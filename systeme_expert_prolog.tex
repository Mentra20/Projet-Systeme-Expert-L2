\documentclass[10pt]{beamer}
% adaptation au français et accents
\usepackage[utf8]{inputenc}
\usepackage[T1]{fontenc}

% insérer des images et graphiques
\usepackage{graphicx}
%video
\usepackage{media9}

%Thème
%\usetheme{Warsaw}
% couleurs
\usepackage{xcolor}
%theme de ses morts LE JOE, LE MARLOT

%pour les numeros de pages
\addtobeamertemplate{footline}{\hspace{1em}\small\insertframenumber/\inserttotalframenumber}
%##########################################################

\begin{document}

\title{\huge Système expert PROLOG}
	\subtitle{\Large L2 Informatique \\
	Déduction interactive d'un animal suivant une base de connaissances }
	\author {Romain Kugler \& Yann Martin D'Escrienne \& Maxime Jerome}
	\institute{\normalsize Université Nice-Sophia Antipolis}
	\date{\today}
	
%Affiche le titre ci-dessus	
\begin{frame} \titlepage \end{frame}

%Intro
\begin{frame}
	\frametitle{\textbf {\Large Introduction}}
	\framesubtitle{\large Contexte}
	
	\begin{block}{}
	Le monde animal contient environ \textbf{8.7 million} d'espèces\\
	\smallskip
	Il convient donc de les classifier selon leurs caractéristiques :
	\medskip
	\begin{itemize}	
		\item Apparence\\
		\item Mode de déplacement\\
		\item Alimentation\\
		\item ect..
	\end{itemize}
	
	\end{block}	
\end{frame}

\begin{frame}
	\frametitle{\textbf {\Large Base du système expert}}
	\framesubtitle{\large Effacement}
	
	\begin{columns}
		\column{0.4\textwidth}
			\begin{itemize}
      
      	%Un système expert est un outil informatique, capable à partir de données d'entrée et par application de règles apprises, de déduire une conclusion logique.
      
      	% Un moteur d'inférence permet aux systèmes experts de conduire des 
        %raisonnements logiques et de dériver des conclusions à partir d'une base de faits et d'une base de connaissances. C'est le progarmme qui réalise cette action
				\item Sert à savoir si la règle est établie à partir des faits
        % moteur d'inférences:  algorithme de simulation des raisonnements déductifs
				\smallskip
				\item Interactif % demande à l'utilisateur d'intéragir
				\smallskip
        % chainage arrière: de la conséquence vers l'antécédants
        % régime par tentatives: Le système dispose d’instructions pour supprimer ou inhiber des faits.
				\item Chainage arrière % c'est lui qui regarde tous les faits à établir via des appels recursifs
				\smallskip
				\item User-Friendly % utilise une communication "Pourquoi" et "Comment"
			\end{itemize}
			
		\column{0.65\textwidth}
			\begin{figure}
				\includegraphics[scale=0.35]{effacer.PNG}
 				\label{pic: Base de connaissance}
 			\end{figure}
	\end{columns}

\end{frame}

\begin{frame}
	\frametitle{\textbf {\Large Base du système expert}}
	\framesubtitle{\large "Comment ?"}
	
	\begin{columns}
		\column{0.4\textwidth}
			\begin{itemize}
				\item La Trace, l'arbre du Comment
				\smallskip
				\item L'arbre est une liste
				\smallskip
				\item Garde les faits effacés
			\end{itemize}
			
		\column{0.65\textwidth}
			\begin{figure}
				\includegraphics[width=\linewidth]{comment.PNG}
 				\label{pic: Base de connaissance}
 			\end{figure}
	\end{columns}

\end{frame}

\begin{frame}
	\frametitle{\textbf {\Large Base du système expert}}
	\framesubtitle{\large "Pourquoi ?"}
	
	\begin{columns}
		\column{0.4\textwidth}
			\begin{itemize}
				\item Explique à l'utilisateur pourquoi le système veut effacer un but
				\smallskip
				\item Est une liste
				\smallskip
				\item Commune pour chaque sous but d'une règle
				\smallskip
				\item La queue est relative à la tête
			\end{itemize}
			
		\column{0.65\textwidth}
			\begin{figure}
				\includegraphics[width=\linewidth]{pourquoi.PNG}
 				\label{pic: Base de connaissance}
 			\end{figure}
 			\begin{figure}
				\includegraphics[width=\linewidth]{pourquoi2.PNG}
 				\label{pic: Base de connaissance}
 			\end{figure}
 			\begin{figure}
				\includegraphics[width=\linewidth]{pourquoi3.PNG}
 				\label{pic: Base de connaissance}
 			\end{figure}
	\end{columns}

\end{frame}



\begin{frame}
	\frametitle{\textbf {\Large Le système expert}}
	\framesubtitle{\large Base de connaissances}
	% ces caractéristiques ont été retranscrites à travers une base de connaissances qui sera interrogée par notre système expert
	\large Organisation de la base de connaissances:
	\bigskip
	\begin{columns}
	
		\column{0.4\textwidth}
		\small
			\begin{itemize}
				\item Animaux \\-> Règles dynamiques
					\smallskip
				\item Animaux \\-> Relation d'arité 1	
				\smallskip
				\item Caractéristiques simples \\-> Faits à déterminer
				\smallskip
				\item Caractéristiques complexes \\-> Règles % qui découlent des caractéristiques simples
			\end{itemize}
			
		\column{0.65\textwidth}
			\begin{figure}
				\includegraphics[width=\linewidth]{SCREEN.PNG}
 				\label{pic: Base de connaissance}
 			\end{figure}
	\end{columns}

\end{frame}

\begin{frame}
	\frametitle{\textbf {\Large Le système expert}}
	\framesubtitle{\large Expertiser : animal}
	
	\begin{columns}
		\column{0.3\textwidth}
			\begin{itemize}
				\item Utilisation de la relation "animal" pour backtrack
				\smallskip
				\item Tentative d'effacement de l'animal 
				\smallskip
				\item Réponse en fonction du résultat d'effacer
			\end{itemize}
			
		\column{0.8\textwidth}
			\begin{figure}
				\includegraphics[scale=0.5]{expertiser.PNG}
 				\label{pic: Base de connaissance}
 			\end{figure}
 			\begin{figure}
				\includegraphics[scale=0.6]{animal.PNG}
 				\label{pic: Base de connaissance}
 			\end{figure}
	\end{columns}

\end{frame}


\begin{frame}
	\frametitle{\textbf {\Large Le système expert }}
	\framesubtitle{\large Élimination des cas}
	
	\bigskip	
	Le système expert ne demande pas la valeur de vérité des animaux ou des règles de bases quand ils échouent. 
			\begin{figure}
				\includegraphics[scale=0.5]{elimination2.PNG}
 				\label{pic: Base de connaissance}
 			\end{figure}
 	
 	Il exclue automatiquement certains cas selon les réponses données.
 	\begin{columns}
	
		\column{0.4\textwidth}
		\begin{figure}
				\includegraphics[scale=0.6]{elimination3.PNG}
 				\label{pic: Base de connaissance}
 		\end{figure}
			
		\column{0.65\textwidth}
			\begin{figure}
				\includegraphics[scale=0.4]{elimination.PNG}
 				\label{pic: Base de connaissance}
 			\end{figure}
	\end{columns}
 			


\end{frame}

\begin{frame}
	\frametitle{\textbf {\Large Démonstration}}
	\framesubtitle{\large Questionnement interactif : Lion}
	\bigskip
	\bigskip
	\bigskip
	 \begin{tikzpicture}[remember picture,overlay]
         \node[anchor=north west, inner sep=30pt] at (current page.north west) {%
           \includemedia[
             addresource=lion.mp4,
             activate=pageopen,transparent,
             flashvars={source=lion.mp4},
             width=0.85\paperwidth,height=0.85\paperheight
           ]{}{VPlayer.swf}%
         };
      \end{tikzpicture}

\end{frame}

\begin{frame}

	\begin{figure}
		\includegraphics[scale=0.17]{frame1.PNG}
 		\label{pic: Base de connaissance}
 	\end{figure}
 	
\end{frame}

\begin{frame}

	\begin{figure}
		\includegraphics[scale=0.17]{frame2.PNG}
 		\label{pic: Base de connaissance}
 	\end{figure}
 	
\end{frame}

\begin{frame}

	\begin{figure}
		\includegraphics[scale=0.17]{frame3.PNG}
 		\label{pic: Base de connaissance}
 	\end{figure}
 	
\end{frame}
\begin{frame}

	\begin{figure}
		\includegraphics[scale=0.17]{frame4.PNG}
 		\label{pic: Base de connaissance}
 	\end{figure}
 	
\end{frame}
\begin{frame}

	\begin{figure}
		\includegraphics[scale=0.17]{frame5.PNG}
 		\label{pic: Base de connaissance}
 	\end{figure}
 	
\end{frame}
\begin{frame}

	\begin{figure}
		\includegraphics[scale=0.17]{frame6.PNG}
 		\label{pic: Base de connaissance}
 	\end{figure}
 	
\end{frame}

\begin{frame}
	\frametitle{\textbf {\Large Démonstration}}
	\framesubtitle{\large Questionnement interactif : abeille}
	\bigskip
	\bigskip
	\bigskip
	 \begin{tikzpicture}[remember picture,overlay]
         \node[anchor=north west, inner sep=30pt] at (current page.north west) {%
           \includemedia[
             addresource=abeille.mp4,
             activate=pageopen,transparent,
             flashvars={source=abeille.mp4},
             width=0.85\paperwidth,height=0.85\paperheight
           ]{}{VPlayer.swf}%
         };
      \end{tikzpicture}
	

\end{frame}

\begin{frame}

	\begin{figure}
		\includegraphics[scale=0.17]{frame1_2.PNG}
 		\label{pic: Base de connaissance}
 	\end{figure}
 	
\end{frame}

\begin{frame}

	\begin{figure}
		\includegraphics[scale=0.17]{frame2_2.PNG}
 		\label{pic: Base de connaissance}
 	\end{figure}
 	
\end{frame}

\begin{frame}

	\begin{figure}
		\includegraphics[scale=0.17]{frame3_2.PNG}
 		\label{pic: Base de connaissance}
 	\end{figure}
 	
\end{frame}

\begin{frame}

	\begin{figure}
		\includegraphics[scale=0.17]{frame4_2.PNG}
 		\label{pic: Base de connaissance}
 	\end{figure}
 	
\end{frame}
\begin{frame}

	\begin{figure}
		\includegraphics[scale=0.17]{frame5_2.PNG}
 		\label{pic: Base de connaissance}
 	\end{figure}
 	
\end{frame}
\begin{frame}

	\begin{figure}
		\includegraphics[scale=0.17]{frame6_2.PNG}
 		\label{pic: Base de connaissance}
 	\end{figure}
 	
\end{frame}

\begin{frame}
	\frametitle{\textbf {\Large Conclusion}}
	
	\begin{itemize}
		
		\item Prototype interactif et explicatif fonctionnel
		\smallskip
		\item Pseudo classification des animaux %application
		\smallskip
		\item Base de connaissance extensible %limites
	\end{itemize}
\end{frame}

%% Rajouter une autre frame pour l'ouverture/limite/questionnement ? 

\end{document}
